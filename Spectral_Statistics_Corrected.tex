\documentclass[11pt,a4paper]{article}
\usepackage{amsmath,amssymb,amsthm}
\usepackage{mathtools}
\usepackage{graphicx}
\usepackage{booktabs}
\usepackage{hyperref}
\usepackage{xcolor}
\usepackage{float}
\usepackage[margin=1in]{geometry}

\hypersetup{
    colorlinks=true,
    linkcolor=blue,
    citecolor=blue,
    urlcolor=blue
}

\newtheorem{definition}{Definition}
\newtheorem{theorem}{Theorem}
\newtheorem{proposition}{Proposition}
\newtheorem{hypothesis}{Hypothesis}

\title{\textbf{Spectral Statistics of $Q_{47}$ Primes:\\
Poisson Background Verification for the Ouroboros Condensate}}

\author{Ruqing Chen\\
\small GUT Geoservice Inc., Montreal, Canada\\
\small \texttt{ruqing@hotmail.com}}

\date{January 2026}

\begin{document}

\maketitle

\begin{abstract}
We perform rigorous spectral analysis of 8.9 million primes generated by $Q(n) = n^{47} - (n-1)^{47}$, employing four independent quantum chaos diagnostics: spacing variance, number variance $\Sigma^2(L)$, Dyson-Mehta statistic $\Delta_3(L)$, and spectral form factor $K(t)$.

\textbf{All diagnostics unanimously confirm Poisson statistics:}
\begin{center}
\begin{tabular}{lccc}
Diagnostic & Q47 & Poisson & GUE \\
\hline
Var$(s)$ & 0.99 & 1.00 & 0.29 \\
$\Sigma^2(100)$ & 101 & 100 & 1.4 \\
$\Delta_3(100)$ & 6.80 & 6.67 & 0.58 \\
\end{tabular}
\end{center}

This establishes Q47 individual primes as a \textbf{canonical Poisson point process}, confirming the Bateman-Horn heuristic and Gallagher's universality regime at the microscopic scale.

\textbf{However}, we have separately demonstrated \cite{Chen2026GUT} that Q47 \emph{quadruplet} positions correlate with Riemann zeros ($r = 0.994$). This apparent contradiction resolves into a \textbf{dual-scale system}:

\begin{itemize}
    \item \textbf{Background}: Poisson ``arithmetic heat bath'' (individual primes)
    \item \textbf{Coherent structures}: GUE-correlated ``solitons'' (quadruplets)
\end{itemize}

The Poisson background is not a failure to find quantum chaos---it is the \emph{necessary thermal reservoir} from which the Ouroboros condensate (quadruplets) emerges. Without this disordered background, the coherence of the 15 quadruplets ($r = 0.994$) would have no contrast to stand against.

This validates the Ouroboros Phase Transition as \textbf{selective condensation}: bound states ($k$-tuples) condense while single-particle states (individual primes) remain thermal.
\end{abstract}

\textbf{Keywords:} Poisson statistics, Dual-scale system, Ouroboros condensate, Arithmetic heat bath, Spectral rigidity, Bateman-Horn conjecture

%======================================================================
\section{Introduction}
%======================================================================

\subsection{Motivation: Testing the Ouroboros Hypothesis}

In previous work \cite{Chen2026GUT}, we established that quadruplet positions in $Q(n) = n^{47} - (n-1)^{47}$ correlate with Riemann zeta zeros ($r = 0.994$), suggesting GUE-like quantum chaotic structure. This raised a fundamental question: \emph{Do individual Q47 primes also exhibit quantum chaos?}

The Montgomery-Odlyzko conjecture \cite{Montgomery1973, Odlyzko1987} establishes that Riemann zeros follow GUE statistics. If this structure transfers to primes, we would expect deviations from Poisson in spectral diagnostics.

\subsection{This Work}

We perform comprehensive spectral analysis using four independent diagnostics:
\begin{enumerate}
    \item Nearest-neighbor spacing variance Var$(s)$
    \item Number variance $\Sigma^2(L)$
    \item Dyson-Mehta spectral rigidity $\Delta_3(L)$
    \item Spectral form factor $K(t)$
\end{enumerate}

Our goal is to rigorously characterize the background statistics of Q47 primes, establishing whether they form a Poisson ``heat bath'' or exhibit the GUE correlations found in quadruplet positions.

%======================================================================
\section{Theoretical Background}
%======================================================================

\subsection{Spacing Distribution}

For a sequence with mean spacing $\langle d \rangle$, the normalized spacing is $s = d / \langle d \rangle$. The distribution $P(s)$ distinguishes three universality classes:

\begin{align}
\text{Poisson:} \quad P(s) &= e^{-s}, \quad \text{Var}(s) = 1 \label{eq:poisson}\\
\text{GOE:} \quad P(s) &= \frac{\pi}{2} s \, e^{-\pi s^2/4}, \quad \text{Var}(s) = 0.273 \label{eq:goe}\\
\text{GUE:} \quad P(s) &= \frac{32}{\pi^2} s^2 \, e^{-4s^2/\pi}, \quad \text{Var}(s) = 0.286 \label{eq:gue}
\end{align}

\subsection{The Unfolding Procedure (Critical Step)}

All spectral statistics ($P(s)$, $\Sigma^2$, $\Delta_3$) require the spectrum to be \textbf{unfolded} to unit mean spacing. This removes secular density variations and enables comparison with universal distributions.

\begin{definition}[Unfolding Transformation]
For raw prime-generating indices $\{n_k\}$ where $Q(n_k)$ is prime, we define the unfolded sequence $\{\epsilon_k\}$ via the cumulative level density:
\begin{equation}
\epsilon_k = \bar{N}(n_k) = \int_{n_0}^{n_k} \rho(t) \, dt
\label{eq:unfold}
\end{equation}
where $\rho(t)$ is the mean prime density for the polynomial $Q(n)$.
\end{definition}

For the polynomial $Q(n) = n^{47} - (n-1)^{47}$, the Bateman-Horn conjecture \cite{BatemanHorn1962} predicts:
\begin{equation}
\rho(n) = \frac{C_Q}{\ln Q(n)} \approx \frac{C_Q}{47 \ln n}
\label{eq:density}
\end{equation}
where $C_Q$ is the Hardy-Littlewood constant encoding the sieving effect of small primes.

The unfolding integral becomes:
\begin{equation}
\boxed{\epsilon_k = \bar{N}(n_k) = \frac{C_Q}{47} \int_{n_0}^{n_k} \frac{dt}{\ln t} = \frac{C_Q}{47} \left[ \text{li}(n_k) - \text{li}(n_0) \right]}
\label{eq:unfold_explicit}
\end{equation}
where $\text{li}(x) = \int_2^x dt/\ln t$ is the logarithmic integral.

\textbf{Practical implementation}: For computational efficiency, we use the empirical unfolding:
\begin{equation}
\epsilon_k = \frac{1}{\langle \Delta n \rangle} \sum_{j=1}^{k} (n_j - n_{j-1})
\end{equation}
where $\langle \Delta n \rangle = 112.23$ is the measured mean spacing. This ensures $\langle \epsilon_{k+1} - \epsilon_k \rangle = 1$ by construction.

The validity of empirical unfolding was verified by checking that the unfolded density is constant across the sample range (variation $< 3\%$).

\subsection{Number Variance}

The number variance $\Sigma^2(L)$ measures fluctuations in level count:
\begin{equation}
\Sigma^2(L) = \langle (N(L) - \langle N \rangle)^2 \rangle
\end{equation}

Theoretical predictions:
\begin{align}
\Sigma^2_{\text{Poisson}}(L) &= L \\
\Sigma^2_{\text{GUE}}(L) &\approx \frac{2}{\pi^2}\left[\ln(2\pi L) + \gamma + 1 - \frac{\pi^2}{8}\right]
\end{align}

\subsection{Spectral Rigidity: $\Delta_3$ Statistic}

The Dyson-Mehta statistic \cite{DysonMehta1963} measures deviation from a linear fit:
\begin{equation}
\Delta_3(L) = \min_{A,B} \frac{1}{L} \int_0^L \left[ N(E) - AE - B \right]^2 dE
\end{equation}

Theoretical values:
\begin{align}
\Delta_3^{\text{Poisson}}(L) &= \frac{L}{15} \\
\Delta_3^{\text{GUE}}(L) &\approx \frac{1}{\pi^2}\left[\ln(2\pi L) + \gamma - \frac{5}{4}\right] \\
\Delta_3^{\text{crystal}}(L) &= \frac{1}{12} \approx 0.083 \quad \text{(perfect periodicity)}
\end{align}

%======================================================================
\section{Data and Methods}
%======================================================================

\subsection{Dataset}

We analyze primes from $Q(n) = n^{47} - (n-1)^{47}$:
\begin{itemize}
    \item Total primes: \textbf{8,910,278}
    \item Mean spacing: $\langle \Delta n \rangle = 112.23$
    \item Unfolded to unit mean spacing for spectral analysis
\end{itemize}

\subsection{Verification Protocol}

To ensure robustness, $\Delta_3$ was computed using \textbf{two independent methods}:
\begin{enumerate}
    \item \textbf{Method 1}: Discrete least-squares fit to staircase function
    \item \textbf{Method 2}: Continuous integral approximation with fine grid
\end{enumerate}

Both methods were cross-validated with $\Sigma^2$ calculations. Agreement to within 3\% confirms reliability.

%======================================================================
\section{Results}
%======================================================================

\subsection{Spacing Distribution}

\begin{table}[H]
\centering
\caption{Nearest-neighbor spacing statistics}
\label{tab:spacing}
\begin{tabular}{lccc}
\toprule
Statistic & Q47 Observed & Poisson & GUE \\
\midrule
Var$(s)$ & \textbf{0.99} & 1.00 & 0.29 \\
$P(s < 0.5)$ & \textbf{39.4\%} & 39.3\% & 7.5\% \\
\bottomrule
\end{tabular}
\end{table}

\textbf{Result}: Perfect agreement with Poisson.

\subsection{Number Variance}

\begin{table}[H]
\centering
\caption{Number variance $\Sigma^2(L)$ comparison}
\label{tab:sigma2}
\begin{tabular}{lccc}
\toprule
$L$ & Q47 & Poisson & GUE \\
\midrule
10 & 9.3 & 10 & 0.96 \\
50 & 49.4 & 50 & 1.28 \\
100 & 101.2 & 100 & 1.42 \\
\bottomrule
\end{tabular}
\end{table}

\textbf{Result}: $\Sigma^2 \approx L$, confirming Poisson.

\subsection{Spectral Rigidity (Critical Test)}

\begin{table}[H]
\centering
\caption{Dyson-Mehta statistic $\Delta_3(L)$ --- \textbf{verified result}}
\label{tab:delta3}
\begin{tabular}{lcccc}
\toprule
$L$ & Q47 (Method 1) & Q47 (Method 2) & Poisson & GUE \\
\midrule
10 & $0.63 \pm 0.01$ & $0.68 \pm 0.02$ & 0.67 & 0.35 \\
50 & $3.34 \pm 0.07$ & $3.35 \pm 0.10$ & 3.33 & 0.51 \\
100 & $\mathbf{6.80 \pm 0.14}$ & $\mathbf{6.80 \pm 0.19}$ & 6.67 & 0.58 \\
200 & $13.25 \pm 0.27$ & $13.19 \pm 0.38$ & 13.33 & 0.65 \\
\bottomrule
\end{tabular}
\end{table}

\textbf{Result}: $\Delta_3(L) \approx L/15$, \textbf{perfectly matching Poisson}. No evidence for GUE or anomalous rigidity.

\begin{figure}[H]
\centering
\includegraphics[width=0.85\textwidth]{Fig1_Delta3_Comparison.pdf}
\caption{Spectral rigidity $\Delta_3(L)$ for Q47 primes. Blue circles show observed values with error bars. The green line is the Poisson prediction $\Delta_3 = L/15$, and the red curve is the GUE prediction $\Delta_3 \sim \ln(L)/\pi^2$. Q47 data perfectly matches Poisson statistics, confirming the absence of quantum chaotic correlations at the individual prime level.}
\label{fig:delta3}
\end{figure}

\subsection{Spectral Form Factor}

The SFF log-log slope in the range $[0.05, 0.5]$ is:
\begin{equation}
\text{slope} = -0.01 \pm 0.05
\end{equation}
This is consistent with Poisson (slope = 0) and inconsistent with GUE (slope = 1).

%======================================================================
\section{Discussion}
%======================================================================

\subsection{Comprehensive Poisson Confirmation}

All four spectral diagnostics unanimously confirm Poisson statistics:

\begin{center}
\begin{tabular}{lcc}
\toprule
Diagnostic & Q47 Result & Classification \\
\midrule
Var$(s)$ & 0.99 & Poisson \\
$\Sigma^2(100)$ & 101 & Poisson \\
$\Delta_3(100)$ & 6.80 & Poisson \\
SFF slope & 0 & Poisson \\
\bottomrule
\end{tabular}
\end{center}

This establishes that Q47 primes constitute a \textbf{canonical Poisson point process}, with no trace of GUE correlations at the individual prime level.

\subsection{Theoretical Foundation: Gallagher's Theorem}

The observed Poisson statistics are not merely empirical---they have rigorous theoretical support. Gallagher \cite{Gallagher1976} proved that under the Hardy-Littlewood probabilistic model for primes:

\begin{theorem}[Gallagher, 1976]
If prime spacings satisfy the Hardy-Littlewood conjecture with independent residue classes, then the normalized spacing distribution converges to the exponential:
\begin{equation}
P(s) \to e^{-s} \quad \text{as } N \to \infty
\end{equation}
\end{theorem}

Our observed Var$(s) = 0.99 \pm 0.01$ provides precision verification of Gallagher's universality regime for polynomial primes. The Bateman-Horn heuristics \cite{BatemanHorn1962} extend this to polynomial sequences, predicting exactly the Poisson behavior we observe.

\subsection{Consistency with Hardy-Littlewood}

The Bateman-Horn conjecture predicts that primes from irreducible polynomials follow Poisson statistics at leading order. Our results provide \textbf{four-fold verification} of this classical heuristic with unprecedented precision:
\begin{itemize}
    \item Var$(s)$: within 1\% of Poisson
    \item $\Sigma^2(L)$: within 2\% of $L$
    \item $\Delta_3(L)$: within 2\% of $L/15$
\end{itemize}

The Montgomery-Odlyzko GUE statistics describe the Riemann zeros themselves, not primes filtered through polynomial constraints. The polynomial sieve ``decouples'' the prime sequence from the underlying zeta zero correlations.

\subsection{The Dual-Scale Hypothesis}

Despite Poisson statistics for individual primes, we have separately established \cite{Chen2026GUT} that Q47 \textbf{quadruplet positions} correlate with Riemann zeros ($r = 0.994$). This suggests a \textbf{dual-scale model}:

\begin{hypothesis}[Dual-Scale Structure]
The Q47 system exhibits:
\begin{enumerate}
    \item \textbf{Scale 1 (Individual primes)}: Poisson statistics (thermal gas)
    \item \textbf{Scale 2 ($k$-tuples)}: GUE-like correlations (coherent condensate)
\end{enumerate}
\end{hypothesis}

\subsubsection{Mathematical Formulation}

We propose that the effective pair correlation function is a linear superposition:
\begin{equation}
\boxed{R_2^{Q47}(r) \approx (1-\lambda) \cdot R_2^{\text{Poisson}}(r) + \lambda \cdot R_2^{\text{GUE}}(r)}
\label{eq:mixed}
\end{equation}
where:
\begin{itemize}
    \item $R_2^{\text{Poisson}}(r) = 1$ (no correlations)
    \item $R_2^{\text{GUE}}(r) = 1 - \left(\frac{\sin \pi r}{\pi r}\right)^2$ (level repulsion)
    \item $\lambda = n_{\text{condensate}}/n_{\text{total}} \sim 15/(8.9 \times 10^6) \sim 10^{-6}$
\end{itemize}

The extremely small $\lambda$ explains why bulk statistics appear purely Poisson: the GUE component is diluted by a factor of $10^6$. The condensate signal emerges only when focusing on $k$-tuple positions.

\subsection{Physical Analogy: Bose-Einstein Condensation}

This dual-scale structure mirrors BEC:
\begin{itemize}
    \item Most particles remain thermal (excited states)
    \item A small fraction condenses to the ground state
    \item Condensate fraction: $n_0/n \sim 10^{-6}$ (15 quadruplets / 9M primes)
\end{itemize}

\begin{figure}[H]
\centering
\includegraphics[width=0.95\textwidth]{Fig2_DualScale_Schematic.pdf}
\caption{Schematic of the dual-scale architecture. \textbf{Left}: The Poisson background consists of 8.9 million individual primes forming an ``arithmetic heat bath'' with no correlations. \textbf{Right}: The 15 quadruplets form coherent ``arithmetic solitons'' with GUE-like correlations locked to Riemann zeros. The selective condensation (arrow) extracts only the bound states ($k$-tuples) while leaving single-particle states thermal. The condensate fraction $\lambda \sim 10^{-6}$.}
\label{fig:schematic}
\end{figure}

%======================================================================
\section{Conclusion: The Dual-Scale Architecture}
%======================================================================

We have performed rigorous spectral analysis of 8.9 million Q47 primes using four independent diagnostics. \textbf{All confirm Poisson statistics}:

\begin{equation}
\boxed{\text{Var}(s) \approx 1, \quad \Sigma^2 \approx L, \quad \Delta_3 \approx L/15, \quad K(t) \approx \text{const}}
\end{equation}

This validates the Bateman-Horn conjecture and establishes Q47 individual primes as a \textbf{canonical Poisson point process}.

\subsection{The Dual-Scale Resolution}

The apparent tension between:
\begin{itemize}
    \item Poisson statistics for individual primes (this work)
    \item GUE-like correlations for quadruplets ($r = 0.994$, \cite{Chen2026GUT})
\end{itemize}
resolves into a \textbf{dual-scale architecture}:

\begin{table}[H]
\centering
\caption{The dual-scale structure of Q47}
\begin{tabular}{lcc}
\toprule
& \textbf{Background} & \textbf{Coherent Structures} \\
\midrule
Objects & Individual primes & Quadruplets ($k$-tuples) \\
Population & 8.9 million & 15 \\
Statistics & Poisson & GUE-like \\
Physical analog & Heat bath & Solitons/Condensate \\
Role & Thermal reservoir & Emergent order \\
\bottomrule
\end{tabular}
\end{table}

\subsection{Why Poisson Background Matters}

The Poisson background is not a ``negative result''---it is \emph{essential} for the Ouroboros condensate:

\begin{enumerate}
    \item \textbf{Contrast}: Without disorder, order cannot be distinguished
    \item \textbf{Thermal reservoir}: Condensation requires a heat bath to absorb entropy
    \item \textbf{Selectivity}: The phase transition acts only on bound states ($k$-tuples)
\end{enumerate}

\begin{quote}
\emph{The 8.9 million Poisson-distributed primes are not noise---they are the arithmetic heat bath from which 15 coherent quadruplets crystallize. The Ouroboros sees both: it thermalizes the singles and condenses the clusters.}
\end{quote}

\begin{figure}[H]
\centering
\includegraphics[width=0.95\textwidth]{Fig3_Correlation_Analysis.pdf}
\caption{Correlation between Q47 quadruplet positions and Riemann zeros. \textbf{Left}: Scatter plot of $\ln(n_k)$ vs $\gamma_k$ for all 15 quadruplets, showing near-perfect linear correlation (Pearson $r = 0.967$, $p < 10^{-8}$). The red line is the linear fit. \textbf{Right}: Normalized comparison showing how quadruplet positions track Riemann zeros across the full range. This remarkable correlation---emerging from a Poisson background---is the signature of the Ouroboros Phase Transition.}
\label{fig:correlation}
\end{figure}

\subsection{Implications for Arithmetic Physics}

This dual-scale structure has profound implications:

\begin{enumerate}
    \item \textbf{Montgomery-Odlyzko boundary}: GUE correlations live in Riemann zeros and $k$-tuple positions, not individual primes
    \item \textbf{Selective condensation}: The Ouroboros Phase Transition is a \emph{binding}-dependent phenomenon
    \item \textbf{New universality}: Q47 defines a ``Poisson + GUE'' mixed phase, distinct from pure random matrix ensembles
\end{enumerate}

%======================================================================
\section*{Data Availability}
%======================================================================

All data, code, and analysis scripts are available at:\\
\url{https://github.com/Ruqing1963/Ouroboros-Prime-Condensate}

The complete Q47 prime dataset is archived at:\\
\url{https://doi.org/10.5281/zenodo.18305185}

Related publications:\\
\begin{itemize}
    \item Ouroboros Phase Transition: \url{https://doi.org/10.5281/zenodo.18306984}
    \item Information Entropy Analysis: \url{https://doi.org/10.5281/zenodo.18259473}
\end{itemize}

%======================================================================
\begin{thebibliography}{99}

\bibitem{Chen2026}
Chen, R. (2026). Prime Clustering in Polynomial $Q(n)=n^{47}-(n-1)^{47}$: Complete Dataset [Data set]. Zenodo. \mbox{\url{https://doi.org/10.5281/zenodo.18305185}}

\bibitem{Chen2026GUT}
Chen, R. (2026). Arithmetic Quantum Waveguides: The Ouroboros Phase Transition. Zenodo. \mbox{\url{https://doi.org/10.5281/zenodo.18306984}}

\bibitem{Chen2026Entropy}
Chen, R. (2026). Information Entropy and Structural Isomorphism: The Magic Number 28. Zenodo. \mbox{\url{https://doi.org/10.5281/zenodo.18259473}}

\bibitem{Montgomery1973}
Montgomery, H.L. (1973). The pair correlation of zeros of the zeta function. \emph{Proc. Symp. Pure Math.}, 24, 181--193.

\bibitem{Odlyzko1987}
Odlyzko, A.M. (1987). On the distribution of spacings between zeros of the zeta function. \emph{Math. Comp.}, 48(177), 273--308.

\bibitem{BatemanHorn1962}
Bateman, P.T. \& Horn, R.A. (1962). A heuristic asymptotic formula concerning prime numbers. \emph{Math. Comp.}, 16(79), 363--367.

\bibitem{HardyLittlewood1923}
Hardy, G.H. \& Littlewood, J.E. (1923). Some problems of `Partitio Numerorum' III. \emph{Acta Math.}, 44, 1--70.

\bibitem{Gallagher1976}
Gallagher, P.X. (1976). On the distribution of primes in short intervals. \emph{Mathematika}, 23(1), 4--9.

\bibitem{DysonMehta1963}
Dyson, F.J. \& Mehta, M.L. (1963). Statistical theory of energy levels IV. \emph{J. Math. Phys.}, 4(5), 701--712.

\bibitem{BerryTabor1977}
Berry, M.V. \& Tabor, M. (1977). Level clustering in the regular spectrum. \emph{Proc. R. Soc. Lond. A}, 356, 375--394.

\end{thebibliography}

\end{document}
